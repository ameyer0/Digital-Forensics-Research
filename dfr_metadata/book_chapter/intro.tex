\section{Introduction}\label{intro}

\begin{paraphrase}
 Both in corporate and government settings, digital forensic (DF) tools are used for post-mortem investigation of cyber-crimes and cyber-attacks. 
Nowadays it is common~\cite{df:news} for law enforcement to use DF tools to follow an electronic trail of evidence to track down a suspect. 
To maintain the quality and integrity of DF tools, National Institute of Standards and Technology (NIST)'s 
Computer Forensics Tool Testing Program (CFTT)~\cite{cftt:nist} 
defined expectations for these tools. Maintaining the standards of DF tools 
is especially critical for judicial proceedings: usage of a forensic tool that does not follow the standards may cause evidence to be thrown 
out in a court case, whereas incorrect results from a forensic tool can also lead improper prosecution of an innocent defendant. 

The focus of this paper is about standardization of one class of DF 
tools that are for Deleted File Recovery (DFR)~\cite{meta:dfr:standards}. 
As the name suggests, a DFR tool attempts to retrieve deleted files
from a file system of a computer. As an example, given a hard disk or a USB drive 
(which might have been seized from the suspect computer or collected from the crime scene), a 
forensics professional can use a DFR tool to investigate about (and potentially retrieve) files which 
a suspect deleted to hide important information. 
The success or failure of a DFR tool can decide the outcome of a case.  

DFR tools are typically classified as one of two varieties, corresponding to two different approaches to file recovery.
These varieties are \emph{metadata-based} tools and \emph{file carving} tools.
The focus of this paper is metadata-based DFR tools, with file carving left for future work.
In the rest of the paper, unless otherwise mentioned, by \emph{DFR tool} we mean metadata-based DFR tool.

Our experiments with a popular DF tool suite named Autopsy~\cite{autopsy} 
show that it does not meet all NIST expectations for DFR. 
Furthermore, we extensively experimented with other frequently used DFR tools. 
We compare those tools' performance and compile a comparative analysis, which could help the user choose the right DFR tool. 

Evaluating the performance of a DFR tool is complex because many elements of a forensics scenario determine 
the success or failure of the file recovery process. 
A few such factors are the type of the file system (FAT, NTFS, etc.), presence of other active/deleted 
files in the file system, fragmentation of a file, a deleted file being overwritten by another file, and so on.
So, comparison of two DFR tools is scientific only if they are compared while keeping these factors same. 
Via extensive analysis, we design a set of test file system images (for either of FAT and NTFS) which considers each of the above factors independently. 
We claim that this list of test cases cover most of the scenarios (except few edge cases) in real-life, and thus claim that our evaluation has broad coverage. 

As there are many file systems (e.g., ext4, HFS, etc.) in addition to FAT and NTFS, one might be interested to know why we chose FAT and NTFS for the current work. 
Because FAT and NTFS are very widely used on external storage devices and devices running Microsoft Windows, respectively,
real-life forensics investigation often involves these two file systems.
While we leave other file systems for future study, our current methodology could be 
extended to other file systems to make a similar study.

The main contributions of the paper are listed as follows:
\begin{itemize}
\item We design and build a list of canonical test file system (FAT and NTFS) images to evaluate the DFR tool per NIST guidelines. 
\item We perform evaluation of frequently-used DFR tools (including free tools as well as proprietary ones)\footnote{As the subject of our evaluation study, 
we have chosen a few tools or trade names. In no case does such choosing imply recommendation or endorsement by the authors.} on the test images.
\item For the interesting cases of tools' success or failure, we provide logical explanation.
\item We provide critique on applicability of some of the NIST guidelines in a practical setting. 
\end{itemize}


The NIST CFTT portal currently publishes reports of only a subset of DFR tools. 
However, that set needs to be expanded as many new tools come to market and become popular.
Also, existing DFR tools should be retested to ensure their reliability is consistent 
as new patches and features come out. 
Adding new reports to the CFTT website will allow tool developers a 
chance to continually develop their tools for the better. We will submit our study reports to the CFTT portal.
At the time of writing, the portal publishes reports for Autopsy~\cite{dhs:autopsy} and FTK~\cite{dhs:ftk}; 
however, both reports are from 2014 and evaluate now-outdated versions of the software.

As a side benefit, our work leads to a few hands-on lab-modules to be used in digital forensics courses 
at BGSU, enriching the new DF specialization program in the CS department. We will also make these modules
publicly available to be used by relevant instructors at other universities.

\end{paraphrase} 
