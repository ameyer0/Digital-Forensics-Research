\section{Finding Summary}
\label{sec:value}

We provide the community a well-labeled Android malware dataset
that includes comprehensive profile information about malware behaviors
and techniques. The dataset provides an up-to-date picture of
the current landscape of Android malware. 
% Our study strongly suggests that \genome dataset is
% outdated as it does not capture most of the current malware
% behaviors. Our observations warrant an urgent need for a new dataset to better
% guide the design of new Android anti-virus solutions.
Below we highlight the main findings we obtained from analyzing the
malware samples in our dataset.

\begin{enumerate}[leftmargin=0cm,itemindent=.5cm,labelwidth=\itemindent,labelsep=0cm,align=left]

\item Android malware monetizing techniques are evolving.
% To better understand the landscape of Android malware,
We need to understand the way cybercriminals generate revenue
for more effective malware analysis and prevention.
% The malware design is dictated by how to steal money 
% and how to keep the malware alive longer,
% so if we perform the analysis along this line, we can
% get more clear picture of the Android malware world.

\item We observe that \emph{standalone} malware apps
are becoming dominant. This indicates that the Android security community
did a good job of capturing repackaged malware, but it also
gives us a message that the war against Android malware has escalated to another stage,
where cybercriminals are becoming more skilled and putting
more effort into designing more comprehensive and sophisticated malware.
% Android ecosystem is growing fast with more and more features being added
% and more financial related activities becoming available.
% This means that Android malware will not stop evolving, so
% anti-virus solutions also have to catchup to guard against the threats.

\item Malware writers are more aggressively using persistence techniques in the malware design.
% Assuming mobile device users care about the security issues of their device,
% any suspicious behavior will alarm the users about existence of a malware.
% So to make the malware stay longer on the infected device, leaving less evidence
% is the key of success for a malware app.

\item Anti-analysis techniques are being widely used in Android malware these days,
and the obfuscation techniques make malware analysts job much harder. % The
% dynamic analysis evading techniques are used to hide the malicious behavior from the
% dynamic vetting sandbox.
When the detection of malware
becomes harder, the malware gets larger time window to exist on the wild harming victims.
This shows an urgent need of designing effective deobfuscation solutions. The 
dynamic analysis tools also need to keep up with evading techniques.
%  obfuscate more its environment to make it more
% like a normal device.

\end{enumerate}

%%% Local Variables: 
%%% mode: latex
%%% TeX-master: "paper"
%%% End: 
