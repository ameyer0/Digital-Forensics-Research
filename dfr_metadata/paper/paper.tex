%
% It is an example file showing how to use the 'acm_proc_article-sp.cls' V3.2SP
% LaTeX2e document class file for Conference Proceedings submissions.

\documentclass{mcurcsm}
%\documentclass{acm_proc_article-sp}

\usepackage[letterpaper,breaklinks=true]{hyperref}
%\usepackage{breakurl}
\usepackage{balance}    % to balance the last page's columns

\begin{document}

    \title{NIST Standards Compliance of Metadata-Based Deleted-File-Recovery (DFR) Tools}
%   \subtitle{[And a sample article]
%   \titlenote{A full version of this paper is available as
%   \textit{Author's Guide to Preparing ACM SIG Proceedings Using
%   \LaTeX$2_\epsilon$\ and BibTeX} at
%   \url{http://www.acm.org/eaddress.htm}}}

%
% You need the command \numberofauthors to handle the 'placement
% and alignment' of the authors beneath the title.
%
% For aesthetic reasons, we recommend 'three authors at a time'
% i.e. three 'name/affiliation blocks' be placed beneath the title.
%
% NOTE: You are NOT restricted in how many 'rows' of
% "name/affiliations" may appear. We just ask that you restrict
% the number of 'columns' to three.
%
% Because of the available 'opening page real-estate'
% we ask you to refrain from putting more than six authors
% (two rows with three columns) beneath the article title.
% More than six makes the first-page appear very cluttered indeed.
%
% Use the \alignauthor commands to handle the names
% and affiliations for an 'aesthetic maximum' of six authors.
% Add names, affiliations, addresses for
% the seventh etc. author(s) as the argument for the
% \additionalauthors command.
% These 'additional authors' will be output/set for you
% without further effort on your part as the last section in
% the body of your article BEFORE References or any Appendices.

\numberofauthors{2} 

\author{
% You can go ahead and credit any number of authors here,
% e.g. one 'row of three' or two rows (consisting of one row of three
% and a second row of one, two or three).
%
% The command \alignauthor (no curly braces needed) should
% precede each author name, affiliation/snail-mail address and
% e-mail address. Additionally, tag each line of
% affiliation/address with \affaddr, and tag the
% e-mail address with \email.
%
% 1st. author
\alignauthor
Andrew Meyer\\
       \affaddr{Bowling Green State University}\\
       \affaddr{Bowling Green, Ohio, USA}\\
%       \affaddr{Address continued}\\
       \email{apmeyer@bgsu.edu}
% 2nd. author
\alignauthor
Sankardas Roy\\
       \affaddr{Bowling Green State University}\\
       \affaddr{Bowling Green, Ohio, USA}\\
%       \affaddr{Address continued}\\
       \email{sanroy@bgsu.edu}
% 3rd. author
%\alignauthor 
%Third Author\titlenote{You can use a ``titlenote'' to recognize your advisor.}\\
%       \affaddr{Affiliation}\\
%       \affaddr{Address}\\
%       \affaddr{Address continued}\\
%       \email{third.author@email.invalid}
%
%\and  % use '\and' if you need 'another row' of author names
%
}

% There's nothing stopping you putting the seventh, eighth, etc.
% author on the opening page (as the 'third row') but we ask,
% for aesthetic reasons that you place these 'additional authors'
% in the \additional authors block, viz.
%\additionalauthors{Additional authors: John Smith (The Th{\o}rv{\"a}ld Group,
%email: {\texttt{jsmith@affiliation.org}}) and Julius P.~Kumquat
%(The Kumquat Consortium, email: {\texttt{jpkumquat@consortium.net}}).}
%\date{30 July 1999}
% Just remember to make sure that the TOTAL number of authors
% is the number that will appear on the first page PLUS the
% number that will appear in the \additionalauthors section.

\maketitle
\begin{abstract}
Digital forensics (DF) tools are used for post-mortem investigation of cyber-crimes and cyber-attacks. 
National Institute of Standards and Technology (NIST) 
has set standards for DF tools. Compliance of the standards by the DF tools is critical, especially in judical proceedings. 
In this paper we consider standardization of one class of DF tools that are for Deleted File Recovery (DFR). 
Our experiments with a popular DF tool suite (named Autopsy) show that it does not satisfy some of the standards for DFR. 
Furthermore, we evaluate other frequently used DFR tools on the same standards, and report the findings.  
We compile a comparative analysis of these tools' performance, which could help the user choose the right DFR tool. 
%We hope our findings are useful to the researcher community as well as the practitioners. 
\end{abstract}


\section{Introduction}

Both in corporate and government settings, digital forensic (DF) tools are used for post-mortem investigation of cyber-crimes and cyber-attacks. 
Nowadays it is common \cite{df:news} for the police to use DF tools to follow an eletronic trail of evidence to track down the suspect. 
To maintain the quality and integrity of DF tools, National Institute of Standards and Technology (NIST)'s 
Computer Forensics Tool Testing Program (CFTT) \cite{cftt:nist} 
set standards for these tools. Maintaining the standards of DF tools 
is especially critical for judicial proceedings: usage of a forensic tool that does not follow the standards may cause evidence to be thrown 
out in a court case whereas incorrect results from a forensic tool can also lead improper prosecution of an innocent defendant. 

The focus of this paper is about standardization of one class of DF 
tools that are for Deleted File Recovery (DFR) \cite{meta:dfr:standards}. 
As the name suggests, a DFR tool attempts to retrieve deleted files
from a file system of a computer. As an example, given a hard disk or a USB drive 
(which might have been seized from the suspect computer or collected from the crime scene), a 
forensics professional can use a DFR tool to investigate about (and potentially retrieve) deleted files that 
a suspect may have deleted to hide important information. 
The success or failure of a DFR tool can decide the outcome of a case.  

DFR tools are typically classified as one of two varieties, corresponding to two different approaches to file recovery.
These varieties are \emph{metadata-based} tools and \emph{file carving} tools.
The focus of this paper is metadata-based DFR tools, with file carving left for future work.
In the rest of the paper, unless otherwise mentioned, by \emph{DFR tool} we mean metadata-based DFR tool.

Our experiments with a popular DF tool suite named Autopsy~\cite{autopsy} 
show that it does not satisfy all NIST standards for DFR. 
Furthermore, we extensively experimented with other frequently used DFR tools. 
We compare those tools' performance and compile a comparative analysis, which could help the user choose the right DFR tool. 

Evaluating the performance of a DFR tool is complex because many elements of a forensics scenario determine 
the success or failure of the file recovery process. 
A few such factors are the type of the file system (FAT, NTFS, etc.), presence of other active/deleted 
files in the file system, fragmentation of a file, a deleted file being overwritten by another file, and so on.
So, comparison of two DFR tools is scientific only if they are compared while keeping these factors same. 
Via extensive analysis, we design a set of test file system images (for either of FAT and NTFS) which considers each of the above factors independently. 
We claim that this list of test cases is exhaustive and thus claim that our evaluation gives a complete picture. 

As there are many file systems (e.g., ext4, HFS, etc.) in addition to FAT and NTFS, one might be interested to know why we chose FAT and NTFS for the current work. 
Because FAT and NTFS are very widely used on external storage devices and devices running Microsoft Windows, respectively,
real-life forensics investigation often involves these two file systems.
While we leave other file systems for future study, our current methodology could be 
used on other file systems to make a similar study.

The main contributions of the paper are listed as follows:
\begin{itemize}
\item We design and build an exhaustive list of canonical test file system (FAT and NTFS) images to test the DFR tool per NIST standards. 
\item We perform evaluation of frequently-used DFR tools (including free tools as well as proprietary ones) on the test images.
\item For the interesting cases of tools' success or failure, we provide logical explanation.
\item We provide critique on applicability of some of the NIST standards in a practical setting. 
\end{itemize}


The NIST CFTT portal currently publishes reports of only a subset of DFR tools. 
However, the scope of tools needs to be expanded as many new tools come to market and become popular.
Also, existing DFR tools should be retested to ensure their reliability is consistent 
as new patches and features come out. 
Adding new reports to the CFTT website will allow tool developers a 
chance to continually develop their tool for the better. We will submit our study reports to the CFTT portal.

As a side benefit, our work leads to a few hands-on lab-modules to be used in digital forensics courses 
at BGSU, enriching the new DF specialization program in the CS department. We will also make these modules
available for relevant instructors at other universities.

\section{Background}

\subsection{Metadata-Based Deleted File Recovery}

\subsection{FAT Filesystem}

\subsection{NTFS Filesystem}

\subsection{NIST Standards}
\begin{enumerate} % TODO cite standards document
 \item ``The tool shall identify all deleted File System-Object entries accessible in residual metadata.''
 \item ``The tool shall construct a Recovered Object for each deleted File System-Object entry accessible in residual metadata.''
 We consider a tool passing this standard as long as it outputs a file for each deleted file, even if the output file is empty.
 \item ``Each Recovered Object shall include all non-allocated data blocks identified in a residual metadata entry.'' For FAT filesystems, we consider a tool passing this standard if it recovers at least the first contiguous segment of unallocated sectors starting from the first sector originally allocated to the deleted file. For NTFS filesystem, the tool must recover all unallocated sectors originally allocated to the deleted file.
 \item ``Each Recovered Object shall consist only of data blocks from the Deleted Block Pool.''
 We consider a tool passing this standard if all recovered sectors were all originally allocated to the deleted file, and had not been reallocated to any other file.
\end{enumerate}


%%% To balance the columns, we need this to be in the code for the LEFT column
% on the last page
\balance

\section{Approach}

\subsection{Creating Test Images}

\subsection{Recovering Files}

\subsection{Results}


\section{Discussion}

\subsection{Ambiguity in Standards}
% TODO Ambiguity of DFR-CR-03 for FAT filesystems

% TODO Cases where 03 and 04 are incompatible (case 5i, etc)

\section{Conclusion}

%Papers should be no more than ten pages, single- spaced, with 12pt font
%for the text content. Left, right, top and bottom margins need to be 1''.  Pages
%should not be numbered (they will be numbered for the proceedings). The title
%and author information are single-column centered at the top of the first page.   
%
%The remaining document should be double- column format. Please give an address
%and email address for each author of the paper. Sections (if used) should be
%numbered and titled. Figures, tables, and other diagrams should be numbered and
%titled. Please try to include each figure near the text it addresses. If
%necessary, you may elect to place large figures at the end of your paper using
%the full width of the page (single-column format). 
%
%Follow the rules of accepted grammar and mathematical formatting. Remember that
%the quality of your paper is one of the criteria for judgment. 

%\subsection{References}
%Use the standard CACM format for references -- that is, a numbered list at the
%end of the article, ordered alphabetically by first author, and referenced by
%numbers in brackets~\cite{bowman:reasoning}.  The references are also in 12pt,
%but that section is ragged right.  Other references that may prove helpful
%include~\cite{Dupre:1995:BW,Lamport:LaTeX,Mittelbach:2004:LC}.
%
%\subsection{Page Number, Headers and Footers}
%Do not include headers, footers, or page numbers in your submission.  These will
%be added when the proceedings are assembled.

%ACKNOWLEDGMENTS are optional
\section{Acknowledgments} This work has been partially supported by BGSU CURS grant in summer of 2019. And a NIST grant.
This section is optional; it is a location for you to acknowledge grants,
funding, editing assistance and what have you.  In the present case, for
example, the authors would like to thank the maintainers of ACM's conference
proceeding \LaTeX\ template from which this document is directly derived and the
past organizers of the MCURCSM conference.

%
% The following two commands are all you need in the
% initial runs of your .tex file to
% produce the bibliography for the citations in your paper.
\bibliographystyle{plain}
\bibliography{mcurcsm-sp}  % mcurcsm-sp.bib is the name of the Bibliography in this case
% You must have a proper ".bib" file
%  and remember to run:
% latex bibtex latex latex
% to resolve all references
%


% That's all folks!
\end{document}
