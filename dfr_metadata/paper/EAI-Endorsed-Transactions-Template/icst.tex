% icstdoc.tex V1.12 6 May 2011

\documentclass[fonts]{icst}

\usepackage{moreverb}

\usepackage[breaklinks,colorlinks,bookmarksopen,bookmarksnumbered,linkcolor=ICSTblue,citecolor=blue,urlcolor=ICSTblue]{hyperref}
\usepackage{breakurl}
\usepackage{doi}


\newcommand\BibTeX{{\rmfamily B\kern-.05em \textsc{i\kern-.025em b}\kern-.08em
T\kern-.1667em\lower.7ex\hbox{E}\kern-.125emX}}

\def\volumeyear{2019}

\journalname{XXXXXX}
\articletype{Research Article/Editorial}
\setcounter{page}{01}
 \def\volumeyear{XXXX}
 \def\volumenumber{XX}
  \def\issuemonth{XX-XX}
  \def\issuenumber{X}
  \def\articleid{XX}
  \def\copyrightholder{Author Name\textit{ et al.}, licensed to ICST}
  \copyrightnote{This is an open access article distributed under the terms of the Creative Commons Attribution license (\url{http://creativecommons.org/licenses/by/3.0/}), which permits unlimited use, distribution and reproduction in any medium so long as the original work is properly cited.}
  \def\articledoi{10.4108/XX.X.X.XX}
\received{XXXX}
  \accepted{XXXX}
  \published{XXXX}

\begin{document}

\runningheads{A. Smith, J.R. Wakeling}{A demonstration of the \journalabb\ class file}

\title{A demonstration of the \LaTeXe\ class file for
\itshape{\journalnamelc}\tnoteref{1}}

\author{Alistair Smith\affil{1}, Joseph Rushton Wakeling\affil{2}\fnoteref{1}}

\address{\affilnum{1}Sunrise Setting Ltd, 12a Fore Street, St. Marychurch, Torquay, Devon, TQ1~4NE, UK\\
\affilnum{2}Institute for Computer Sciences, Social Informatics and Telecommunications Engineering (ICST)}

\abstract{This paper describes the use of the \LaTeXe\
\textsf{\journalclass} class file for setting papers for
submission to \emph{\journalnamelc}.}

\keywords{class file, \LaTeXe, \emph{\journalabb}}

\tnotetext[1]{Please ensure that you use the most up to date
class file, available from EAI at \url{http://doc.eai.eu/publications/transactions/latex/}
}

\fnotetext[1]{Corresponding author.  Email: \email{publications@eai.eu}}

\maketitle

\section{Introduction}
Many authors submitting to research journals use \LaTeXe\ to
prepare their papers. This paper describes the
\textsf{\journalclass} class file which can be used to convert
articles produced with other \LaTeXe\ class files into the correct
form for publication in \emph{\journalnamelc}.

The \textsf{\journalclass} class file preserves much of the
standard \LaTeXe\ interface so that any document which was
produced using the standard \LaTeXe\ \textsf{article} style can
easily be converted to work with the \textsf{\journalclassshort}
style. However, the width of text and typesize will vary from that
of \textsf{article.cls}; therefore, \emph{line breaks will change}
and it is likely that displayed mathematics and tabular material
will need re-setting.

In the following sections we describe how to lay out your code to
use \textsf{\journalclass} to reproduce the typographical look of
\emph{\journalnamelc}. However, this paper is not a guide to
using \LaTeXe\ and we would refer you to any of the many books
available (see, for example, \cite{R1,R2,R3}).

\section{The three golden rules}
Before we proceed, we would like to stress \emph{three golden
rules} that need to be followed to enable the most efficient use
of your code at the typesetting stage:
\begin{enumerate}
\item[(i)] keep your own macros to an absolute minimum;

\item[(ii)] as \TeX\ is designed to make sensible spacing
decisions by itself, do \emph{not} use explicit horizontal or
vertical spacing commands, except in a few accepted (mostly
mathematical) situations, such as \verb"\," before a
differential~d, or \verb"\quad" to separate an equation from its
qualifier;

\item[(iii)] follow the \emph{\journalnamelc} reference style.
\end{enumerate}

\section{Getting started}
The \textsf{\journalclassshort} class file should run
on any standard \LaTeXe\ installation. If any of the fonts, style
files or packages it requires are missing from your installation,
they can be found on the \emph{\TeX\ Collection} DVDs or from
CTAN.

\emph{\journalnamelc} are published using a combination of Kp-Fonts
and Iwona typefaces. This is achieved by using the \verb"fonts"
option as\\
\verb"\documentclass[fonts]{icst}".

\noindent If for any reason you have a problem using this
combination you can easily resort to Computer Modern fonts by
removing the \verb"fonts" option.

\begin{figure*}
\setlength{\fboxsep}{0pt}%
\setlength{\fboxrule}{0pt}%
\begin{center}
\begin{boxedverbatim}
\documentclass[fonts]{icst}
%\documentclass[fonts,doublespace]{icst}%For paper submission

\begin{document}

\runningheads{<Initials and Surnames>}{<Short title>}

\title{<Initial caps>}
%\title{<Initial caps>\tnoteref{1}}

\author{<An Author\affil{1}, Someone Else\affil{2}\fnoteref{1},
Perhaps Another\affil{1}>}

\address{<\affilnum{1}First author's address
(in this example it is the same as the third author)\\
\affilnum{2}Second author's address>}

%\tnotetext[1]{<Footnote to the title, if needed>}

\fnotetext[1]{Corresponding author. \email{<email address>}}
%Corresponding author is the second author in this example

%\fnotetext[2]{<Text as needed>}

\abstract{<Text>}

\keywords{<List keywords>}

\maketitle

\section{Introduction}
.
.
.
\end{boxedverbatim}
\end{center}
%\vspace{-12pt}
\caption{Example header text\label{F1}}
\end{figure*}


\section{The article header information}
The heading for any file using \textsf{\journalclass} is shown in
Figure~\ref{F1}.

\subsection{Remarks}
\begin{enumerate}
\item[(i)] In \verb"\runningheads" use `\emph{et~al.}' if there
are four or more authors.

\item[(ii)] Note the use of \verb"\tnoteref{<id num>}" and \verb "\tnotetext[<id num>]{<Text>}" for setting footnotes to the title

Also note the use of \verb"\fnoteref{<id num>}" and
\verb"\fnotetext[<id num>]{<Text>}" for setting footnotes to author names. At
least one of these must identify the `Corresponding author' (as in the example shown in Figure~\ref{F1})

\item[(iii)] For submitting a double-spaced manuscript, add
\verb"doublespace" as an option to the documentclass line.

\item[(iv)] The abstract should be capable of standing by itself,
in the absence of the body of the article and of the bibliography.
Therefore, it must not contain any reference citations.

\item[(v)] Keywords are separated by commas.
\end{enumerate}

\begin{figure}
\setlength{\fboxsep}{0pt}%
\setlength{\fboxrule}{0pt}%
\begin{center}
\begin{boxedverbatim}
\begin{table}\small\sf
\caption{<Table caption>}
\centering
\begin{tabular}{<table alignment>}
\toprule
<column headings>\\
\midrule
<table entries
(separated by & as usual)>\\
<table entries>\\
.
.
.\\
\bottomrule
\end{tabular}
\end{table}
\end{boxedverbatim}
\end{center}
\caption{Example table layout\label{F2}}
\end{figure}

\section{The body of the article}

\subsection{Mathematics} \textsf{\journalclass} makes the full
functionality of \AmS\/\TeX\ available. We encourage the use of
the \verb"align", \verb"gather" and \verb"multline" environments
for displayed mathematics. \textsf{amsthm} is used for setting
theorem-like and proof environments. The usual \verb"\newtheorem"
command needs to be used to set up the environments for your
particular document.

\subsection{Figures and Tables} \textsf{\journalclass} includes the
\textsf{graphicx} package for handling figures.

Figures are called in as follows:
\begin{verbatim}
\begin{figure}
\centering
\includegraphics{<figure name>}
\caption{<Figure caption>}
\end{figure}
\end{verbatim}

For further details on how to size figures, etc., with the
\textsf{graphicx} package see, for example, \cite{R1}
or \cite{R3}. If figures are available in an
acceptable format (for example, .eps, .ps) they will be used but a
printed version should always be provided. \medbreak

The standard coding for a table is shown in Figure~\ref{F2}.

\subsection{Cross-referencing}
The use of the \LaTeX\ cross-reference system
for figures, tables, equations, etc., is encouraged
(using \verb"\ref{<name>}" and \verb"\label{<name>}").

\subsection{Appendices}
Code appendices as follows.
\begin{verbatim}
\begin{appendices}
\section{<The first appendix title>}
.
.
.
\subsection{The first subappendix}

\appendix
\section{<The second appendix title>}
.
.
.
\end{appendices}
\end{verbatim}


\subsection{Acknowledgements} An Acknowledgements section is started with \verb"\ack" or
\verb"\acks" for \textit{Acknowledgement} or
\textit{Acknowledgements}, respectively. It must be placed just
before the References.

\subsection{Bibliography}
The \emph{\journalnamelc} allows either the Vancouver or Harvard
reference style. The default style is Vancouver, but the Harvard
style can be achieved easily by using the \verb"authoryear"
option as:\\
\verb"\documentclass[...,authoryear]{icst}".

Please note that the files \textsf{icstnum.bst} and
\textsf{icstauth.bst} (which output references in the Vancouver
and Harvard style, respectively) are available from the same
download page for those authors using \BibTeX.

\subsection{Double Spacing}
If you need to double space your document for submission please
use the \verb+doublespace+ option as shown in the sample layout in
Figure~\ref{F1}.

\section{Support for \textsf{\journalclass}}
We offer on-line support to participating authors. Please contact
us via e-mail at \email{publications@eai.eu}

We would welcome any feedback, positive or otherwise, on your
experiences of using \textsf{\journalclass}.

\section{Copyright statement}
Please  be  aware that the use of  this \LaTeXe\ class file is
governed by the following conditions.

\subsection{Copyright}
The Copyright licensed to ICST.

\ack This class file was developed by Sunrise Setting Ltd,
Torquay, Devon, UK. Website:\\
\href{http://www.sunrise-setting.co.uk}{\texttt{www.sunrise-setting.co.uk}}

\begin{thebibliography}{9}

\bibitem{R1} \textsc{Kopka~H.} and \textsc{Daly~P.W.} (2003) \emph{A Guide to \LaTeX}
(Addison-Wesley), 4th~ed.

\bibitem{R2} \textsc{Lamport~L.} (1994) \emph{\LaTeX: a Document Preparation System}
(Addison-Wesley), 2nd~ed.

\bibitem{R3} \textsc{Mittelbach~F.} and \textsc{Goossens~M} (2004) \emph{The \LaTeX\ Companion} (Addison-Wesley), 2nd~ed.
\end{thebibliography}
\end{document}
