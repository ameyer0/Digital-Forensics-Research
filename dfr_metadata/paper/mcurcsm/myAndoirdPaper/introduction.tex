\vspace{-.1in}
\section{Introduction}
\label{sec:introduction}
\vspace{-.02in}

The Android platform continues to dominate the smartphone market
with more than 80\% share according to the study by 
International Data Corporation~\cite{idc} and Gartner~\cite{gartner}.
Over the last five years, the Android world has been changing dramatically with
more features added, and more sensitive operations (\eg banking and wallet) 
becoming popular on smartphones.
Along with the Android platform's popularity, the Android malware has been growing as well,
with more complex logic and anti-analysis techniques. 
%(such as, dynamic loading\morecite, code obfuscation\morecite).

As expected, research groups across academia and industry put
enormous effort to design novel methods to detect Android malware. 
However, the above effort is adversely affected by the lack of clear understanding of 
the latest Android malware landscape. 
A reliable ground truth dataset is essential for building effective 
malware analysis techniques and verifying the validity of new detection methods.
% In order to design effective detection solutions for new Android malware,
% it is necessary to obtain key insights by studying the recent malware behaviors. 
For understanding the nefarious techniques used in the state-of-the-art malware apps, 
{\bf detailed behavior profiles for each malware variety must be provided
in such a dataset}.
While creating such a dataset is a must-do ground work,
this task is extremely difficult.
In particular,
% Unfortunately, such a dataset is extremely difficulty to construct.
%There are significant challenges in creating a new, more up-to-date dataset for Android malware.
%in studying the overall landscape of recent Android malware.
{\bf to provide the rich information for malware behaviors, manual analysis is indispensable.}
However it is not feasible to manually analyze all Android malware at our hands (we have \samsize
from various sources). Thus the first step is to categorize the samples into
semantically equivalent groups; then we only need to study a few samples
from each group.


One can use AV scanning service like
VirusTotal~\cite{virustotal} to group malware samples into families;
however, the family labels returned are often
inconsistent~\cite{hurier2016lack,sebastian2016avclass}.
% The first challenge is due to scalability issue.
Moreover, we observe that malware samples within one family
may actually contain different varieties with different behaviors.
Thus we cannot simply rely upon the grouping provided by AV products, even after being refined
by tools like AVclass~\cite{sebastian2016avclass}.
% It is infeasible to manually study every sample for
% a large family, \eg when the family size is larger than a few hundred.
% The second challenge is due to difficulty in automating the Android malware categorization task.
% We can use clustering to group malware based on their semantic features.
% Unfortunately, traditional PC malware clustering techniques are not readily
% applicable to Android apps due to their excessive usage of 
% shared libraries and the repackaging techniques.
Even if grouping has been done perfectly, the amount of work of manually analyzing
representative apps from each malware variety is still daunting.
Advanced obfuscation methods are
widely adopted in recent Android malware apps, further complicating the manual
analysis process.



%Nevertheless, 
%the Android research community \morecite is still using the dataset from 

Due to the above reasons, there has not been any effort on creating such a
rich Android malware dataset, except for the 
\genome~\cite{zhou2012dissecting} project.
The \genome dataset is no longer
available to researchers due to resource limitations.
 % (which includes analysis of each family)
 % that the community can use.
It provided a malware dataset containing
1260 malware samples categorized in 49 families,
discovered in 2010 and 2011.
%We experimented with this dataset and observed that
We have collected a more recent Android malware dataset from several 
sources (VirusShare, Google Play and third party security companies).
% while the majority
The malware in this collection were discovered between 2010 and 2016. 
We made comparative study of the Genome dataset with our malware samples of 2011 and later,
and found that the majority of the threats in those newer samples are not captured by the 
Genome samples. As a result, we not only need a more up-to-date malware dataset for Android,
we also need one with much richer semantic information than what the Genome dataset provided.



%\genome data is too outdated to capture some of the new threats.
%The \genome  is old, without up-to-date understanding of Android malware,
%and new data set, practical methods cannot designed to capture the new threats.

%Note that there are significant challenges in studying the overall landscape of recent Android malware and their categorization. 
%One challenge is due to scalability issue while another challenge stems from the fact that traditional technique to handle 
%desktop malware does not readily work for the Android domain for the following reasons:
%In particular, security companies receive hundreds of ``potentially malicious'' Android samples everyday \morecite while 
%traditional desktop malware triage process such as malware clustering is not directly applicable to Android domain for following reasons:
%The excessive usage of libraries in the Android world and the repackaging techniques
%adopted by Android malware authors make clustering process challenging - for instance, two malicious Android app samples sharing 
%high level of overall similarity does not guarantee that they belong to the same family.
%And even among the app samples which are labeled with the same family name by anti-virus companies (\eg listed on Virustotal), 
%there may exist significant varieties (sub-families) of different version of malicious payloads.
%On one hand, malware samples with new version payload may break existing malware detection technique, so it is necessary to analyze each version of malicious samples individually.
%On the other hand, it becomes almost infeasible to manually study every sample for a large family; for instance, when a family size is as large as 8000.


\vspace{.1in}

%Understanding of the those techniques will help to (why/what/how)
\noindent {\bf The main contributions of this work are as follows.}
\begin{enumerate}[label=\textbf{\arabic*}.]
\vspace{-.1in}
\item
We present a systematic method of analyzing large volumes of Android malware samples with high
%confidence, which helps us to prepare a large ground truth Android malware dataset.
%This method addressed the scalability challenge by leveraging a two layer grouping technique,
%(1) Group malware samples with same family name, 
%(2) Refine each family to semantically different varieties using clustering analysis.
%We then perform manual analysis on each variety of each family to generate behavior report.
%The final dataset contains \samsize labeled Android malware samples that are classified in \versize varieties within \fsize families, 
confidence, which helps us prepare a large ground truth Android malware dataset with rich profile information.
This method addresses the scalability challenge by leveraging a two-step grouping technique
followed by a systematic and deep manual analysis.
\item We present a detailed guideline for performing the manual analysis so other researchers can 
replicate the process on other Android malware samples in their
possession.
Our manual analysis provides profiles for each variety of the Android malware
regarding their behaviors.
% , we further conduct 
% a comprehensive study to profile
% their behaviors, and monetization methods.
This provides insights into the landscape of the current Android malware.
\item We prepare a comprehensive dataset which contains \samsize labeled Android malware samples that are classified in \versize 
varieties within \fsize families, 
whose discovery dates range from 2010 to 2016. 
%We collected the samples from various sources, including VirusShare, Google Play, and third party security companies.
We publish detailed reports including behavior information for each malware variety at our
\amd \pubsite.
We are sharing the whole dataset with the research community.
%\item We propose a  
%payload sharing based malware clustering approach to
%separate malware samples from a particular family to different versions.
%For each particular version samples, we randomly select xxx samples 
%to conduct detailed manual analysis and verify the correctness of
%clustering results. 
%For each malware family we using cluster analysis to get the different version,
%and for each version of each malware family we detailed analyze the behavior
%it has and write down the document.
% \vspace{-.02in}
% \item 



% In our dataset, we find that 
% (a) since 2012 the percentage of malware varieties that are standalone is increasing and 
% that of repackaged malware is decreasing.
% % 63\%/35\%\footnote{\% of varieties/\% of apps} are standalone apps as opposed to 30\%/7\% being repackaged 
% % apps\footnote{The remaining are cases where an app developer used a malicious library without knowing it.},
% This indicates that malware writers are recently investing more resource in Android malware development;
% (b) the percentage of malware varieties that gain root privilege is decreasing;
% % Only 24\%/5\% malware families attempt to gain root privilege, which shows
% this indicates that Android OS is becoming harder to exploit;
% however, the recent rooting malware are becoming more sophisticated, 
% which can potentially
% make them as complex as PC malware in the near future;
% (c) more Android malware % 63\%/79\%
% apply anti-analysis techniques to evade detection, which
% indicates the urgent need of advanced de-obfuscation and 
% dynamic analysis tools;
% % within one malware family as well as between different malware families.
% (d) We observe that
% malware are evolving towards monetization.
% % 61\%/86\% adopt at least one monetization method;
% % among them, 52\%/85\% are related to banking, ransom and aggressive advertisement, 
% 61\% of malware varieties adopt at least one monetization method;
% among them, 52\% are related to banking, ransom and aggressive advertisement, 
% which is different from old methods such as subscribing to premium services.
% %this shows that the current malware apps use a diversity of ways to get more revenue.
% %Futhermore, other growing techniques support the goal of generating more revenue.
% %For instance, applying device-admin privilege helps malware gain more control of infected device.
% %The more efficient persistence techniques keep malware operating longer, and
% %the extensive use of anti-analysis techniques makes the malware harder to be detected and removed.
% %(1) Obtaining system-admin privilege has become more popular,
% %because it can give the malware the power to encrypt the file system,
% %modify PIN code, wipe user data; and make it difficult to be uninstalled;
% %(2) Using timer or {\em AlarmManager} to schedule
% %timing task periodically (\eg collecting victim's data and fetching command from the C\&C
% %server) has become more dominant in recent malware design;
% %an extreme use of this is found in recent ransomware apps;
% %(3) Malware design is moving towards using persistence techniques, \eg 
% %evidence-cleaning, prevention-from-destruction methods are invented to keep the malware
% %staying longer on the infected device.
\end{enumerate} 


% \vspace{-.05in}
The rest of the paper is organized as follows. Section~\ref{sec:data} discusses
the process of preparing the dataset.
Section~\ref{sec:profile} discusses in details the behaviors and techniques of malware in our dataset,
and Section~\ref{sec:evolution} discusses our analysis and observation of the malware evolution trends.
%We summarize the main findings of our analysis in Section~\ref{sec:value},
We discuss related research in Section~\ref{sec:related},
and conclude in Section~\ref{sec:conclusion}.

%%% Local Variables: 
%%% mode: latex
%%% TeX-master: "paper"
%%% End: 
