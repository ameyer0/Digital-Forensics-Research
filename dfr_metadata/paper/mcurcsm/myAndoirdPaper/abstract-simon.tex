\begin{abstract}
\vspace{.02 in}
% The Android platform has captured the majority of the smartphone market share. 
% Unfortunately, 
%The phenomenal success of the Android platform attracted large amount of malware.
To build effective malware analysis techniques and to evaluate new detection tools, 
up-to-date datasets reflecting the current Android malware landscape 
are essential. For such datasets to be maximally useful,
they need to contain reliable and complete information on malware's behaviors and
techniques used in the malicious activities. Such a dataset shall also provide a
comprehensive coverage of a large number of types of malware. 
The \genome created circa 2011 has been the only well-labeled and widely studied dataset the
research community had easy access to\footnote{As of 12/21/2015 the 
Genome authors have stopped supporting the dataset sharing due to resource limitation.}.
But not only is it outdated and no longer represents the current Android malware 
landscape, 
it also does not provide as detailed information on malware's behaviors 
as needed for research.
Thus it is urgent to create a high-quality dataset for Android malware. %  to facilitate
% research in this important area.
% The main challenge 
% is to provide accurate and reliable information for the malware behaviors and
% to scale the analysis at the same time.
While existing information sources such as VirusTotal are useful, 
to obtain the accurate and detailed information 
for malware behaviors, deep manual analysis is indispensable. 
In this work we present our approach to preparing a large Android malware dataset
for the research community. We leverage existing anti-virus scan results
and automation techniques in categorizing our large dataset (containing \samsize malware app samples) 
into \versize varieties (based on malware behavioral semantics) which belong to \fsize malware families.
For each variety, we select three samples as representatives, for a total of 405 malware samples,
to conduct in-depth manual analysis.
% leading to the manual analysis of about 400 malware apps that are representatives of the above varieties, and 
% leverage existing tools to manually analyze 
% these representative apps.
Based on the manual analysis result we generate detailed descriptions of each malware variety's behaviors and include them in our dataset. %  using 
% state-of-the art static analysis tools.
We also report our observations on the current landscape of Android malware as depicted in the dataset.
% We produce detailed documentation of both the manual analysis process and
% the results, and 
%Not only do we provide the research community with the new up-to-date dataset, 
Furthermore, we present detailed documentation of the process used in creating the dataset, including
the guidelines for the manual analysis. 
% the results, and
%Our work is the first that presents a systematic process of analyzing 
%a large number of Android malware samples so that this process can be replicated by other researchers 
%on other Android malware samples in their possession.
We make our Android malware dataset available to the research community. 
%The detailed profile and behavior information about each malware variety can be found at the website \url{http://www.androidmalwaredataset.org/}.
%The complete dataset will be shared upon request with a research purpose.

% Traditional techniques such as clustering 
% analysis cannot be directly applied due to the widespread use of shared third-party 
% libraries and repackaging techniques in the Android malware world. In this work, 
% we leverage antivirus scanning results from VirusTotal and threat reports
% from anti-virus community
% to get family names 
% for the collected Android malware samples. We then refine the malware families through a 
% payload-sharing-based clustering analysis augmented with manual verification, 
% and 
%We perform a comprehensive study of behaviors, techniques, and trend of Android malwares,
%and produce a large labeled dataset with \samsize samples of \fsize malware families
%ranging from 2010 to 2016.
%We further perform automated clustering analysis 
%(augmented with occasional manual study) to categorize
%malware samples in several families while one family may have 30 versions.
%Through manual study of representative samples from each version of each family, we
%produce detailed documentation and list our observations to provide research community with a
%landscape of recent Android malware.
%For each malware family, we study its
%(1) versions with timeline
%and geography;
%(2) installation \& activation strategies;
%(3) behaviors \eg privilege escalation, remote control, financial charges,
%personal information stealing, download \& install application, stealthiness;
%(4) anti-analysis \& obfuscation techniques.
% This dataset will help the Android research community to have an up-to-date 
% understanding of the Android malware landscape and produce better anti-virus 
% solutions in the future. We also present the key observations we made for the
% current Android malware landscape.
\end{abstract}
\vspace{-.03 in}

%%% Local Variables: 
%%% mode: latex
%%% TeX-master: "paper"
%%% End: 
