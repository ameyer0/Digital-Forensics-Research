\section{Research Questions}

A DFR tool is a piece of software that can retrieve (residual data of) a file that was deleted 
from a storage device (e.g., computer hard disk, flash drive, or so). We evaluate a set of 
popular DFR tools on the scale of CFTT standards. 
In particular, below are the research questions (RQs) that we target to answer. 

\begin{itemize}
\item[RQ1.] Do the popular DFR tools (as available in the market) meet the NIST CFTT standards? 
If not, which tool meets which part of the standard? What factors make the tools fail or succeed?

\item[RQ2.] Are the free DFR tools more effective compared to the enterprise-level (proprietary) tools?
\end{itemize}

The identification of errors, such as for not recovering a deleted file or attempting to recover a file that was never there 
(Type I and Type II errors, respectively), is an important metric for a DFR tool. 
Type I and Type II errors account for majority of the standard. Many factors impact the performance of a DFR tool, 
including the condition of the file system (e.g., the file system being almost full or empty) and 
how the file was deleted (being placed in Recycle bin, permanent deletion, reformat of disk, or so). 
We consider these variables in the design of experiment when we compare the tools. 
In addition to core features of CFTT standards \cite{cftt:nist}, we also use optional features \cite{cftt:nist} for comparing the DFR tools.

