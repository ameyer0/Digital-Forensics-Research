\section{Research Questions}

A DFR tool is a piece of software that can retrieve residual data of a file that was deleted 
from a storage device (e.g., computer hard disk, flash drive, and so on). We evaluate a set of 
popular DFR tools on the scale of CFTT standards. 
In particular, following are the research questions (RQs) that we target to answer. 

\begin{itemize}
\item[RQ1.] Do the popular DFR tools meet the NIST CFTT standards? 
If not, which tool meets which part of the standard? 

\item[RQ2.] What factors make the tools fail or succeed?

\item[RQ3.] Are the free DFR tools more effective compared to the enterprise-level (proprietary) tools?
\end{itemize}

The identification of errors, such as for not recovering a deleted file or attempting to recover a file that was never there 
(Type I and Type II errors, respectively), is an important metric for a DFR tool. 
Type I and Type II errors account for majority of the standard. Many factors impact the performance of a DFR tool, 
including file system type, whether the file content is not located in contiguous clusters, whether 
some part of the deleted file content is overwritten by another file, and more.
We consider these variables in the design of experiment when we compare the tools.
Note that our current study is limited to exploring the \emph{core features} of NIST standards \cite{meta:dfr:standards}, 
i.e., we leave the optional features \cite{meta:dfr:standards} of NIST standards for future study.

