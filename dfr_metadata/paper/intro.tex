
\section{Introduction}

Both in corporate and government settings, digital forensic (DF) tools are used for post-mortem investigation of cyber-crimes and cyber-attacks. 
National Institute of Standards and Technology (NIST)'s Computer Forensics Tool Testing Program (CFTT) \cite{cftt:nist} 
proposed standards for DF tools to help determine their quality and integrity. Maintaining the standards of DF tools 
is especially critical for judicial proceedings: Usage of a forensic tool that does not follow the standards may cause evidence to be thrown 
out in a court case whereas incorrect results from a forensic tool can also lead improper prosecution of an innocent defendant. 

The focus of our proposed work is about standardization of one class of DF 
tools that are for Deleted File Recovery (DFR). XXXXXXXXXX explain what a DFR tool is with example. 
XXXXXX also mention FAT NTFS in this context.
 
Our experiments with a popular DF tool suite (named Autopsy \cite{autopsy}) show that it does not satisfy all standards for DFR. 
Furthermore, our experiments with other frequently used DFR tools. 
We will also compare those tools' performance and compile a comparative analysis, which could help the user choose the right DFR tool. 
Our work will also result a few hands-on lab-modules for the future students at BGSU, enriching the new DF specialization program in the CS department.

The main contributions of the paper are listed as follows:
\begin{itemize}
\item evaluation of widely used DFR tools (including free tools as well as propritory ones) on FAT and NTFS file systems per NIST standards.
\item reasoning for tools' success or failure
\item critique on applicability of NIST standards in practical setting. 
\end{itemize}
