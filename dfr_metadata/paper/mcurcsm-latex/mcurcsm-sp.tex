%
% It is an example file showing how to use the 'acm_proc_article-sp.cls' V3.2SP
% LaTeX2e document class file for Conference Proceedings submissions.

\documentclass{mcurcsm}
%\documentclass{acm_proc_article-sp}

\usepackage[letterpaper,breaklinks=true]{hyperref}
%\usepackage{breakurl}
\usepackage{balance}    % to balance the last page's columns

\begin{document}

    \title{Guidelines for MCURCSM 2019\titlenote{Modified from the sample
    supplied by ACM as part of their SIG proceedings template
    http://www.acm.org/sigs/publications/proceedings-templates.}}
%   \subtitle{[And a sample article]
%   \titlenote{A full version of this paper is available as
%   \textit{Author's Guide to Preparing ACM SIG Proceedings Using
%   \LaTeX$2_\epsilon$\ and BibTeX} at
%   \url{http://www.acm.org/eaddress.htm}}}

%
% You need the command \numberofauthors to handle the 'placement
% and alignment' of the authors beneath the title.
%
% For aesthetic reasons, we recommend 'three authors at a time'
% i.e. three 'name/affiliation blocks' be placed beneath the title.
%
% NOTE: You are NOT restricted in how many 'rows' of
% "name/affiliations" may appear. We just ask that you restrict
% the number of 'columns' to three.
%
% Because of the available 'opening page real-estate'
% we ask you to refrain from putting more than six authors
% (two rows with three columns) beneath the article title.
% More than six makes the first-page appear very cluttered indeed.
%
% Use the \alignauthor commands to handle the names
% and affiliations for an 'aesthetic maximum' of six authors.
% Add names, affiliations, addresses for
% the seventh etc. author(s) as the argument for the
% \additionalauthors command.
% These 'additional authors' will be output/set for you
% without further effort on your part as the last section in
% the body of your article BEFORE References or any Appendices.

\numberofauthors{3} 

\author{
% You can go ahead and credit any number of authors here,
% e.g. one 'row of three' or two rows (consisting of one row of three
% and a second row of one, two or three).
%
% The command \alignauthor (no curly braces needed) should
% precede each author name, affiliation/snail-mail address and
% e-mail address. Additionally, tag each line of
% affiliation/address with \affaddr, and tag the
% e-mail address with \email.
%
% 1st. author
\alignauthor
First Author\\
       \affaddr{Affiliation}\\
       \affaddr{Address}\\
       \affaddr{Address continued}\\
       \email{first.author@email.invalid}
% 2nd. author
\alignauthor
Second Author\\
       \affaddr{School name}\\
       \affaddr{School address}\\
       \affaddr{Address continued}\\
       \email{second.author@email.invalid}
% 3rd. author
\alignauthor 
Third Author\titlenote{You can use a ``titlenote'' to recognize your advisor.}\\
       \affaddr{Affiliation}\\
       \affaddr{Address}\\
       \affaddr{Address continued}\\
       \email{third.author@email.invalid}
%
%\and  % use '\and' if you need 'another row' of author names
%
}

% There's nothing stopping you putting the seventh, eighth, etc.
% author on the opening page (as the 'third row') but we ask,
% for aesthetic reasons that you place these 'additional authors'
% in the \additional authors block, viz.
%\additionalauthors{Additional authors: John Smith (The Th{\o}rv{\"a}ld Group,
%email: {\texttt{jsmith@affiliation.org}}) and Julius P.~Kumquat
%(The Kumquat Consortium, email: {\texttt{jpkumquat@consortium.net}}).}
%\date{30 July 1999}
% Just remember to make sure that the TOTAL number of authors
% is the number that will appear on the first page PLUS the
% number that will appear in the \additionalauthors section.

\maketitle
\begin{abstract}
Your abstract should be limited to 100 words. 
Summarize the main point(s) of the paper 
focusing on your contribution. Please avoid the 
use of non-text symbols, fonts, and other editing 
devices so that a text version of your abstract can 
be placed on the conference web page paper.
\end{abstract}

\section{Eligibility}
The 2019 Midwest Conference on Undergraduate Research in Computer Science and
Mathematics (MCURCSM'19) is intended for a professional audience of peers and
faculty. Submissions should be from students who are currently enrolled in an
undergraduate program. Students who have just graduated in Spring 2019 may also
submit their research work as long as the work submitted was performed while
the student was enrolled as an undergraduate or during the summer 2019
immediately following their graduation. 

While many or most of the projects will have a 
sponsoring faculty advisor, the student should 
have completed a significant portion of the work 
presented in the paper. The faculty advisor may 
have originated the central idea and provided 
guidance, but it is the student who should have 
primary ownership of the material presented in 
the paper. The writing should also be the 
student's own although the faculty advisor is 
encouraged to provide generous editing support. 
A group of students may also submit a single 
paper. It is acceptable, based on contribution, for 
faculty to appear as co-authors although the 
student should be the primary author. 

\section{Submissions}
\subsection{Paper Content}
The paper should focus on a problem in mathematics, computer science or a
closely related field in which mathematics or computer science is a primary
emphasis of the work. While MCURCSM'19 welcomes all student research
submissions, primary consideration will be given to original research
contributions. In this end, it is important for the student to identify this
facet of their work in the submitted paper. Students should highlight what has
been accomplished beforehand and by whom, and then make it clear how they
advance the state of knowledge with their work. 

\subsection{Review Criteria}
All submitted papers will be subject to a rigorous and professional review
process. Mathematics and computer science faculty will review the papers using
criteria similar to that of a professional conference or journal in their field.
The primary review criteria are: 

\begin{itemize}
    \item The importance of the research to the field of study. 
    \item  The quality of the research. 
    \item  The quality of the paper. 
\end{itemize}

Each submission will be reviewed by multiple 
faculty reviewers and rated according to the above criteria. Papers will be
accepted for the conference by combining the scores of the reviewers. The
conference committee will make final decisions in the event of ties. A comment
section is included in the review process; students will receive this comment
section only. 
Submissions will be acknowledged with one of three outcomes: acceptance,
conditional acceptance, or no acceptance. Papers that are conditionally accepted
will have a list of requirements that the student must correct/change for the
paper to be accepted. Typically these are small or medium-scale editing remarks. 

%%% To balance the columns, we need this to be in the code for the LEFT column
% on the last page
\balance

\section{Paper Format}
You can download a \LaTeX\ style and the source to this document (to use as an
example) or a Microsoft Word template 
from the MCURCSM  web page  
\url{http://mcurcsm.mscs.mu.edu}.
 
Papers should be no more than ten pages, single- spaced, with 12pt font
for the text content. Left, right, top and bottom margins need to be 1''.  Pages
should not be numbered (they will be numbered for the proceedings). The title
and author information are single-column centered at the top of the first page.   

The remaining document should be double- column format. Please give an address
and email address for each author of the paper. Sections (if used) should be
numbered and titled. Figures, tables, and other diagrams should be numbered and
titled. Please try to include each figure near the text it addresses. If
necessary, you may elect to place large figures at the end of your paper using
the full width of the page (single-column format). 

Follow the rules of accepted grammar and mathematical formatting. Remember that
the quality of your paper is one of the criteria for judgment. 

\subsection{References}
Use the standard CACM format for references -- that is, a numbered list at the
end of the article, ordered alphabetically by first author, and referenced by
numbers in brackets~\cite{bowman:reasoning}.  The references are also in 12pt,
but that section is ragged right.  Other references that may prove helpful
include~\cite{Dupre:1995:BW,Lamport:LaTeX,Mittelbach:2004:LC}.

\subsection{Page Number, Headers and Footers}
Do not include headers, footers, or page numbers in your submission.  These will
be added when the proceedings are assembled.

%ACKNOWLEDGMENTS are optional
\section{Acknowledgments}
This section is optional; it is a location for you to acknowledge grants,
funding, editing assistance and what have you.  In the present case, for
example, the authors would like to thank the maintainers of ACM's conference
proceeding \LaTeX\ template from which this document is directly derived and the
past organizers of the MCURCSM conference.

%
% The following two commands are all you need in the
% initial runs of your .tex file to
% produce the bibliography for the citations in your paper.
\bibliographystyle{plain}
\bibliography{mcurcsm-sp}  % mcurcsm-sp.bib is the name of the Bibliography in this case
% You must have a proper ".bib" file
%  and remember to run:
% latex bibtex latex latex
% to resolve all references
%


% That's all folks!
\end{document}
