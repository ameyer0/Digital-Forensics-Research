\section{Conclusion}

We designed an exhaustive set of canonical test cases to determine a metadata-based DFR tool's compliance with the NIST CFTT standards.
We tested five popular DFR tools and evaluated their results.
We present a comparison of the tools based on the number of test cases for which each tool meets the standards.
We conclude that none of the tested tools consistently fulfill the NIST standards, and explain the factors which cause them to fail.
We also identify potential weaknesses in the standards and suggest improvements.

\subsection{Future Work}
\comment{would it make more sense to make this the last subsection of Discussion?}
The NIST CFTT standard includes several optional features; these features could be explored using a similar methodology.
We only created test images for the FAT and NTFS filesystems; our process could be expanded to other common filesystems such as ext4 and HFS.
NIST CFTT has a separate set of standards for file carving DFR tools; future work could involve creating test cases for file carving tools.
